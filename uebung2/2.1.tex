\documentclass{article}
\usepackage[utf8]{inputenc}
\usepackage{ngerman}
\usepackage{amsmath}
\usepackage{amsfonts}
\usepackage[margin=1.2in]{geometry}
\date{}

\begin{document}

\section*{Aufgabe 2.1}

\bf{a)}\normalfont\\
\\
Ein LOOP-Programm besteht aus einer endlichen Aneinanderreihung von Anweisungen.
Wertzuweisungen sind offensichtlich in endlicher Zeit ausführbar. LOOP-Anweisungen überprüfen die übergebene Variable und wiederholen das jeweilige Teilprogramm entsprechend oft.
Da Variablen nur endliche Werte haben können, kann das Teilprogramm auf diese Weise auch nur endlich oft wiederholt werden, es muss also in endlicher Zeit ausführbar sein.
Da also alle Anweisungen eines LOOP-Programmes in endlicher Zeit ausführbar sind, muss jedes LOOP-Programm in endlicher Zeit terminieren.\\
\\
\\
\bf{b)}\normalfont\\
\\
$
P ::=\\
\indent x_0 := 1;\\
\indent \text{WHILE } x_0 \neq 0 \text{ DO } x_0 := x_0 + 1 \text{ END}
$\\
\\
\\
\bf{c)}\normalfont\\
\\
Sei $p$ eine nat"urliche Zahl.\\
\\
LOOP:\\
$
P ::=\\
\indent x_0 := p;\\
\indent \text{LOOP } x_0 \text{ DO } Q \text{ END}
$\\
\\
WHILE:\\
$
P ::=\\
\indent x_0 := p;\\
\indent \text{WHILE } x_0 \neq 0 \text{ DO } Q; x_0 := x_0 - 1 \text{ END}
$\\
\\
\\
\bf{di)}\normalfont\\
\\
$
P ::=\\
\indent result := y+0;\\
\indent x_1 := x+0;\\
\indent \text{WHILE } x_1 \neq 0 \text{ DO }\\
\indent\indent x_1 := x_1 - 1;\\
\indent\indent result := result + 1;\\
\indent\text{END}
$\\
\\
Ergebnis: $result$
\newpage
\noindent
\bf{dii)}\normalfont\\
\\
$
P ::=\\
\indent result := 0;\\
\indent x_{Mult} := x+0;\\
\indent \text{WHILE } x_{Mult} \neq 0 \text{ DO }\\
\indent\indent y_{Add} := y + 0;\\
\indent\indent x_{Mult} := x_{Mult}-1;\\
\indent\indent \text{WHILE } y_{Add} \neq 0 \text{ DO }\\
\indent\indent\indent y_{Add} := y_{Add} -1;\\
\indent\indent\indent result := result + 1;\\
\indent\indent\text{END}\\
\indent\text{END}
$\\
\\
Ergebnis: $result$\\
\\
\\
\bf{diii)}\normalfont\\
\\
$
P ::=\\
\indent result := 0;\\
\indent \text{WHILE } x \neq 0 \text{ DO }\\
\indent\indent result := 1;\\
\indent\indent y_{Pot} := y + 0;\\
\indent\indent \text{WHILE } y_{Pot} \neq 0 \text{ DO }\\
\indent\indent\indent y_{Pot} :=  y_{Pot} - 1;\\
\indent\indent\indent x_{Mult} := x - 1;\\
\indent\indent\indent tempRes := result;\\
\indent\indent\indent \text{WHILE } x_{Mult} \neq 0 \text{ DO }\\
\indent\indent\indent\indent x_{Mult} := x_{Mult} - 1;\\
\indent\indent\indent\indent resAdd := tempRes + 0;\\
\indent\indent\indent\indent \text{WHILE } x_{Mult} \neq 0 \text{ DO }\\
\indent\indent\indent\indent\indent resAdd := resAdd - 1;\\
\indent\indent\indent\indent\indent result := result + 1;\\
\indent\indent\indent\indent\text{END}\\
\indent\indent\indent\text{END}\\
\indent\indent\text{END}\\
\indent\text{END}
$\\
\\
Ergebnis: $result$
\end{document}